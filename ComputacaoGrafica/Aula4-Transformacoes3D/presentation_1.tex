%%%%%%%%%%%%%%%%%%%%%%%%%%%%%%%%%%%%%%%%%
% Beamer Presentation
% LaTeX Template
% Version 1.0 (10/11/12)
%
% This template has been downloaded from:
% http://www.LaTeXTemplates.com
%
% License:
% CC BY-NC-SA 3.0 (http://creativecommons.org/licenses/by-nc-sa/3.0/)
%
%%%%%%%%%%%%%%%%%%%%%%%%%%%%%%%%%%%%%%%%%

%----------------------------------------------------------------------------------------
%	PACKAGES AND THEMES
%----------------------------------------------------------------------------------------

\documentclass{beamer}

\mode<presentation> {

% The Beamer class comes with a number of default slide themes
% which change the colors and layouts of slides. Below this is a list
% of all the themes, uncomment each in turn to see what they look like.

%\usetheme{default}
%\usetheme{AnnArbor}
%\usetheme{Antibes}
%\usetheme{Bergen}
%\usetheme{Berkeley}
%\usetheme{Berlin}
%\usetheme{Boadilla}
%\usetheme{CambridgeUS}
%\usetheme{Copenhagen}
\usetheme{Darmstadt}
%\usetheme{Dresden}
%\usetheme{Frankfurt}
%\usetheme{Goettingen}
%\usetheme{Hannover}
%\usetheme{Ilmenau}
%\usetheme{JuanLesPins}
%\usetheme{Luebeck}
%\usetheme{Madrid}
%*\usetheme{Malmoe}
%\usetheme{Marburg}
%\usetheme{Montpellier}
%\usetheme{PaloAlto}
%\usetheme{Pittsburgh}
%\usetheme{Rochester}
%\usetheme{Singapore}
%\usetheme{Szeged}
%\usetheme{Warsaw}

% As well as themes, the Beamer class has a number of color themes
% for any slide theme. Uncomment each of these in turn to see how it
% changes the colors of your current slide theme.

%\usecolortheme{albatross}
%\usecolortheme{beaver}
%\usecolortheme{beetle}
%\usecolortheme{crane}
%\usecolortheme{dolphin}
%\usecolortheme{dove}
%\usecolortheme{fly}
%\usecolortheme{lily}
\usecolortheme{orchid}
%\usecolortheme{rose}
%\usecolortheme{seagull}
%\usecolortheme{seahorse}
%\usecolortheme{whale}
%\usecolortheme{wolverine}

%\setbeamertemplate{footline} % To remove the footer line in all slides uncomment this line
%\setbeamertemplate{footline}[page number] % To replace the footer line in all slides with a simple slide count uncomment this line

%\setbeamertemplate{navigation symbols}{} % To remove the navigation symbols from the bottom of all slides uncomment this line
}


\usepackage{graphicx} % Allows including images
\usepackage{booktabs} % Allows the use of \toprule, \midrule and \bottomrule in tables
\usepackage{xspace}
\usepackage{caption}
\usepackage{subfigure}
\usepackage[english,brazil]{babel}
\usepackage[utf8]{inputenc}

%Renomeia o nome padrao das figuras.
\renewcommand{\figurename}{Figura}
\renewcommand{\tablename}{Tabela}
%----------------------------------------------------------------------------------------
%	TITLE PAGE
%----------------------------------------------------------------------------------------

\title[Computação Gráfica]{Transformações 3D} % The short title appears at the bottom of every slide, the full title is only on the title page

\author{Uéliton Freitas} % Your name
\institute[UFMS] % Your institution as it will appear on the bottom of every slide, may be shorthand to save space
{
Universidade Católica Don Bosco - UCDB \\ % Your institution for the title page
\medskip
\textit{freitas.ueliton@gmail.com} % Your email address
}
\date{\today} % Date, can be changed to a custom date


\begin{document}

\begin{frame}
\titlepage % Print the title page as the first slide
\end{frame}

\begin{frame}
\frametitle{Sumário} % Table of contents slide, comment this block out to remove it
\tableofcontents % Throughout your presentation, if you choose to use \section{} and \subsection{} commands, these will automatically be printed on this slide as an overview of your presentation
\end{frame}




%----------------------------------------------------------------------------------------
%	PRESENTATION SLIDES
%----------------------------------------------------------------------------------------

%------------------------------------------------
\section{Introdução} 
%------------------------------------------------

%\section{Speeded-Up Robust Features - SURF} % A subsection can be created just before a set of slides with a common theme to further break down your presentation into chunks
%\section{Baf Of Features and Colors}

%\section{Refer\^encias}
%%%%%%%%%%%%%%%%%%%%%%%%%%%%%%%%%%%%%%%%%%%%%%%%%%%%%%%%%%%%%%%%%%%%%%%%%%%%%%%%%%%%%%%%%%
\begin{frame}
\frametitle{Introdução}


	\begin{block}{Transformações 3D}
		\begin{itemize}
			\item<1-> Transformações 3D são extensões das transformações 2D.
			\item<2-> A translação e escala são simples adaptações.
			\item<3-> A rotação é mais complexa.
				\begin{itemize}
					\item Pode-se efetuar a rotação em qualquer eixo espacial. Diferente da rotação 2D que é feita em torno do eixo $x,y$.
				\end{itemize}
			\item<4-> As posições 3D são expressadas por coordenadas homogêneas por meio de matrizes com 4 linhas e colunas, ou seja, as matrizes 3D são $4x4$.
		\end{itemize}
	\end{block}
	
\end{frame}


%%%%%%%%%%%%%%%%%%%%%%%%%%%%%%%%%%%%%%%%%%%%%%%%%%%%%%%%%%%%%%%%%%%%%%%%%%%%%%%%%%%%%%%%%%
\section{Transformações Básicas}
\subsection{Translação}
\begin{frame}
\frametitle{Translação}


	\begin{block}{Translação 3D}
		\begin{itemize}
			\item Um objeto é deslocado adicionando-se um deslocamento a cada uma das direções cartesianas.
			\begin{eqnarray*}
				x' = x + \Delta x \\
				y' = y + \Delta y \\
				z' = z + \Delta z \\
			\end{eqnarray*}
		\end{itemize}
	\end{block}
	
\end{frame}

%%%%%%%%%%%%%%%%%%%%%%%%%%%%%%%%%%%%%%%%%%%%%%%%%%%%%%%%%%%%%%%%%%%%%%%%%%%%%%%%%%%%%%%%%%
\begin{frame}
\frametitle{Translação}

	\begin{block}{Representação Matricial da Translação 3D}
		
			\begin{eqnarray*}
				P' = T(\Delta x,\Delta y,\Delta z) \cdot P\\
				\begin{bmatrix}
					x'	\\
					y'	\\
					z'	\\
					1
				\end{bmatrix}								
				= \begin{bmatrix}
					1	&	0	&	0	&	\Delta x	\\
					0	&	1	&	0	&	\Delta y	\\
					0	&	0	& 	1	&	\Delta z	\\
					0	&	0	&	0	&	1		\\
				\end{bmatrix}
				\cdot \begin{bmatrix}
					x	\\
					y	\\
					z	\\
					1
				\end{bmatrix}
			\end{eqnarray*}
	
	\end{block}

	
	
\end{frame}

%%%%%%%%%%%%%%%%%%%%%%%%%%%%%%%%%%%%%%%%%%%%%%%%%%%%%%%%%%%%%%%%%%%%%%%%%%%%%%%%%%%%%%%%%%
\begin{frame}
\frametitle{Translação}

	\begin{figure}[!h]
			\begin{center}
			\includegraphics[width=0.5\textwidth]{Figures/translacao3D}
			\end{center}
	\end{figure}	
\end{frame}


%%%%%%%%%%%%%%%%%%%%%%%%%%%%%%%%%%%%%%%%%%%%%%%%%%%%%%%%%%%%%%%%%%%%%%%%%%%%%%%%%%%%%%%%%%
\begin{frame}
\frametitle{Translação Inversa}

	\begin{block}{Representação Matricial da Translação 3D Inversa}
		
			\begin{eqnarray*}
				T^{-1}(\Delta x,\Delta y,\Delta z) =  T(- \Delta x,- \Delta y, -\Delta z)\\
			T^{-1}(\Delta x,\Delta y,\Delta z) = \begin{bmatrix}
					1	&	0	&	0	&	-\Delta x	\\
					0	&	1	&	0	&	-\Delta y	\\
					0	&	0	& 	1	&	-\Delta z	\\
					0	&	0	&	0	&	1		\\
				\end{bmatrix}
			\end{eqnarray*}
	\end{block}
\end{frame}

%%%%%%%%%%%%%%%%%%%%%%%%%%%%%%%%%%%%%%%%%%%%%%%%%%%%%%%%%%%%%%%%%%%%%%%%%%%%%%%%%%%%%%%%%%
\subsection{Escala 3D}
\begin{frame}
\frametitle{Escala}

	\begin{block}{Escala 3D}
		\begin{itemize}
			\item A escala 3D é uma simples extensão da escala 2D.
			\begin{eqnarray*}
				s_x > 0, s_y > 0,s_z > 0 \\
				x' = x \cdot s_x \\
				y' = y \cdot s_y \\
				z' = z \cdot s_z \\
			\end{eqnarray*}
		\end{itemize}
	\end{block}
\end{frame}


%%%%%%%%%%%%%%%%%%%%%%%%%%%%%%%%%%%%%%%%%%%%%%%%%%%%%%%%%%%%%%%%%%%%%%%%%%%%%%%%%%%%%%%%%%
\begin{frame}
\frametitle{Escala}

	\begin{block}{Representação Matricial da Escala 3D}
		\begin{itemize}
			\item A escala 3D é uma simples extensão da escala 2D.
			\begin{eqnarray*}
				P' = S(s_x,s_y,s_z) \cdot P\\
				\begin{bmatrix}
					x'	\\
					y'	\\
					z'	\\
					1
				\end{bmatrix}								
				= \begin{bmatrix}
					s_x	&	0	&	0	&	0	\\
					0	&	s_y	&	0	&	0	\\
					0	&	0	& 	s_z	&	0	\\
					0	&	0	&	0	&	1	\\
				\end{bmatrix}
				\cdot \begin{bmatrix}
					x	\\
					y	\\
					z	\\
					1
				\end{bmatrix}
			\end{eqnarray*}
		\end{itemize}
	\end{block}
\end{frame}


%%%%%%%%%%%%%%%%%%%%%%%%%%%%%%%%%%%%%%%%%%%%%%%%%%%%%%%%%%%%%%%%%%%%%%%%%%%%%%%%%%%%%%%%%%
\begin{frame}
\frametitle{Translação}

	\begin{block}{Escala 3D}
		\begin{itemize}
			\item Com $s_x > 1$ e $s_y > 1$
				\begin{itemize}
					\item O objeto se afasta da origem.
				\end{itemize}
				
			\item Com $ 0 <s_x < 1$ e $ 0 < s_y < 1$
				\begin{itemize}
					\item O objeto se aproxima da origem.
				\end{itemize}
		\end{itemize}
	\end{block}


	\begin{figure}[!h]
			\begin{center}
			\includegraphics[width=0.5\textwidth]{Figures/Escala3D}
			\end{center}
	\end{figure}	
\end{frame}

%%%%%%%%%%%%%%%%%%%%%%%%%%%%%%%%%%%%%%%%%%%%%%%%%%%%%%%%%%%%%%%%%%%%%%%%%%%%%%%%%%%%%%%%%%
\begin{frame}
\frametitle{Translação}

	\begin{block}{Escala 3D - Ponto Fixo}
		\begin{itemize}
			\item Para resolver o problema do deslocamento:
				\begin{itemize}
					\item Translada-se o ponto fixo do objeto para a origem.
					\item Efeuta-se a escala.
					\item Translada-se o ponto fixo para a posição original.\\
					$P' =T(\Delta x_f, \Delta y_f, \Delta z_f) \cdot S(s_x,s_y,s_z) \cdot T(- \Delta x_f, - \Delta y_f, - \Delta z_f)$
					\begin{eqnarray*}
						\begin{bmatrix}
							x'	\\
							y'	\\
							z'	\\
							1
						\end{bmatrix}								
				= \begin{bmatrix}
					s_x	&	0	&	0	&	(1-s_x)x_f	\\
					0	&	s_y	&	0	&	(1-s_y)y_f	\\
					0	&	0	& 	s_z	&	(1-s_z)z_f	\\
					0	&	0	&	0	&	1	\\
				\end{bmatrix}
				\cdot \begin{bmatrix}
					x	\\
					y	\\
					z	\\
					1
				\end{bmatrix}
			\end{eqnarray*}
			\end{itemize}
		\end{itemize}
	\end{block}
\end{frame}


%%%%%%%%%%%%%%%%%%%%%%%%%%%%%%%%%%%%%%%%%%%%%%%%%%%%%%%%%%%%%%%%%%%%%%%%%%%%%%%%%%%%%%%%%%
\begin{frame}
\frametitle{Escala Inversa}

	\begin{block}{Escala Inversa 3D}
		\begin{itemize}
			\item A matriz da escala inversa é obtida trocando os valores das escalas por seus valores inversos.
			\begin{eqnarray*}
				T^{-1}(s_x,s_y,s_z) = \begin{bmatrix}
					\frac{1}{s_x}	&	0	&	0	&	(1-\frac{1}{s_x})x_f	\\
					0	&	\frac{1}{s_y}	&	0	&	(1-\frac{1}{s_y})y_f	\\
					0	&	0	& 	\frac{1}{s_z}	&	(1-\frac{1}{s_z})z_f	\\
					0	&	0	&	0	&	1	\\
				\end{bmatrix}
			\end{eqnarray*}
		\end{itemize}
	\end{block}
\end{frame}


%%%%%%%%%%%%%%%%%%%%%%%%%%%%%%%%%%%%%%%%%%%%%%%%%%%%%%%%%%%%%%%%%%%%%%%%%%%%%%%%%%%%%%%%%%
\subsection{Rotação 3D}
\begin{frame}
\frametitle{Rotação 3D}

	\begin{block}{Rotação 3D}
		\begin{itemize}
			\item Pode-se rotacionar um objeto em qualquer eixo no espaço 3D.
			\item Contudo, é mais fácil fazer a rotação sobre os eixos cartesianos.
				\begin{itemize}
					\item É possível combinar qualquer rotação em torno dos eixos cartesianos para obter rotações em qualquer eixo no espaço.
				\end{itemize}
			\item Ângulos positivos rotacionam o objeto no sentido anti-horário.
		\end{itemize}
	\end{block}
\end{frame}

%%%%%%%%%%%%%%%%%%%%%%%%%%%%%%%%%%%%%%%%%%%%%%%%%%%%%%%%%%%%%%%%%%%%%%%%%%%%%%%%%%%%%%%%%%
\begin{frame}
\frametitle{Orientação da Rotação 3D}

	\begin{figure}[!h]
			\begin{center}
			\includegraphics[width=0.7\textwidth]{Figures/OriRotacao}
			\end{center}
	\end{figure}
	
\end{frame}


%%%%%%%%%%%%%%%%%%%%%%%%%%%%%%%%%%%%%%%%%%%%%%%%%%%%%%%%%%%%%%%%%%%%%%%%%%%%%%%%%%%%%%%%%%
\begin{frame}
\frametitle{Orientação da Rotação 3D}

	\begin{block}{Rotação 3D}
		\begin{itemize}
			\item A rotação 3D pode ser facilmente estendida da rotação 2D.
			\begin{eqnarray*}
				x' = x \cdot cos \theta - y \cdot sen \theta \\
				y' = x \cdot sen \theta - y \cdot cos \theta \\
				z' = z
			\end{eqnarray*}
		\end{itemize}
	\end{block}
	
	\begin{block}{Rotação 3D na Representação Matricial}
		\begin{itemize}
			\item A rotação 3D pode ser facilmente estendida da rotação 2D.
			\begin{eqnarray*}
				\textbf{P}' = \textbf{R}(\theta) \cdot \textbf{P}\\
				\begin{bmatrix}
					x'	\\
					y'	\\
					z'	\\
					1
				\end{bmatrix}
				= \begin{bmatrix}
					cos \theta	& -sen \theta	& 0 & 0 \\
					sen \theta	& cos \theta		& 0 & 0 \\
					0			& 0				& 1 & 0 \\
					0			& 0				& 0	& 1
				\end{bmatrix}
				\cdot \begin{bmatrix}
					x	\\
					y	\\
					z	\\
					1
				\end{bmatrix}
			\end{eqnarray*}
		\end{itemize}
	\end{block}
	
\end{frame}

%%%%%%%%%%%%%%%%%%%%%%%%%%%%%%%%%%%%%%%%%%%%%%%%%%%%%%%%%%%%%%%%%%%%%%%%%%%%%%%%%%%%%%%%%%
\begin{frame}
\frametitle{Rotação no eixo $z$}

	\begin{figure}[!h]
			\begin{center}
			\includegraphics[width=0.7\textwidth]{Figures/Rz}
			\end{center}
	\end{figure}
	
\end{frame}


%%%%%%%%%%%%%%%%%%%%%%%%%%%%%%%%%%%%%%%%%%%%%%%%%%%%%%%%%%%%%%%%%%%%%%%%%%%%%%%%%%%%%%%%%%
\begin{frame}
\frametitle{Rotação 3D}

	\begin{block}{Rotação 3D}
		\begin{itemize}
			\item A rotação entre os eixos podem ser feitas por meio de uma permutação cíclica das coordenadas $x,y$ e $z$.\\
			$x \to y \to z \to x$
		\end{itemize}
	\end{block}

	\begin{figure}[!h]
			\begin{center}
			\includegraphics[width=0.9\textwidth]{Figures/Rcirc}
			\end{center}
	\end{figure}
	
\end{frame}

%%%%%%%%%%%%%%%%%%%%%%%%%%%%%%%%%%%%%%%%%%%%%%%%%%%%%%%%%%%%%%%%%%%%%%%%%%%%%%%%%%%%%%%%%%
\begin{frame}
\frametitle{Rotação 3D}

	\begin{block}{Rotação 3D no eixo $x$}
		\begin{itemize}
			\item Levando em consideração a permutação citada, a rotação no eixo $x$ é composta da seguinte equação:\\
			\begin{eqnarray*}
				\begin{bmatrix}
					x' \\
					y' \\
					z' \\
					1
				\end{bmatrix} = 
				\begin{bmatrix}
					1			& 0				& 0 & 0 \\
					cos \theta	& -sen \theta	& 0 & 0 \\
					sen \theta	& cos \theta		& 0 & 0 \\
					0			& 0				& 0	& 1
				\end{bmatrix}
				\cdot \begin{bmatrix}
					x \\
					y \\
					z \\
					1
				\end{bmatrix}
			\end{eqnarray*}
		\end{itemize}
	\end{block}
	
\end{frame}

%%%%%%%%%%%%%%%%%%%%%%%%%%%%%%%%%%%%%%%%%%%%%%%%%%%%%%%%%%%%%%%%%%%%%%%%%%%%%%%%%%%%%%%%%%
\begin{frame}
\frametitle{Rotação 3D}


	\begin{figure}[!h]
			\begin{center}
			\includegraphics[width=0.7\textwidth]{Figures/rx}
			\caption{Rotação em torno do eixo $x$.}
			\end{center}
	\end{figure}
	
\end{frame}

%%%%%%%%%%%%%%%%%%%%%%%%%%%%%%%%%%%%%%%%%%%%%%%%%%%%%%%%%%%%%%%%%%%%%%%%%%%%%%%%%%%%%%%%%%
\begin{frame}
\frametitle{Rotação 3D}

	\begin{block}{Rotação 3D no eixo $y$}
		\begin{itemize}
			\item Levando em consideração a permutação citada, a rotação no eixo $y$ é composta da seguinte equação:\\
			\begin{eqnarray*}
				\begin{bmatrix}
					x' \\
					y' \\
					z' \\
					1
				\end{bmatrix} = 
				\begin{bmatrix}
					cos \theta	& sen \theta		& 0 & 0 \\
					0			& 1				& 0 & 0 \\
					-sen \theta	& cos \theta	& 0 & 0 \\
					0			& 0				& 0	& 1
				\end{bmatrix}
				\cdot \begin{bmatrix}
					x \\
					y \\
					z \\
					1
				\end{bmatrix}
			\end{eqnarray*}
		\end{itemize}
	\end{block}
	
\end{frame}

%%%%%%%%%%%%%%%%%%%%%%%%%%%%%%%%%%%%%%%%%%%%%%%%%%%%%%%%%%%%%%%%%%%%%%%%%%%%%%%%%%%%%%%%%%
\begin{frame}
\frametitle{Rotação 3D}


	\begin{figure}[!h]
			\begin{center}
			\includegraphics[width=0.7\textwidth]{Figures/ry}
			\caption{Rotação em torno do eixo $y$.}
			\end{center}
	\end{figure}
	
\end{frame}

%%%%%%%%%%%%%%%%%%%%%%%%%%%%%%%%%%%%%%%%%%%%%%%%%%%%%%%%%%%%%%%%%%%%%%%%%%%%%%%%%%%%%%%%%%
\begin{frame}
\frametitle{Rotação 3D}

	\begin{block}{Inversa da Rotação 3D}
		\begin{itemize}
			\item A inversa de uma rotação em $\theta$, é dada fazendo a rotação em $-\theta$.
			\item Como o que muda é apenas o sinal do seno, a inversa também pode ser obtida por $\textbf{R}^{-1} = \textbf{R}(-\theta)$.
		\end{itemize}
	\end{block}
	
\end{frame}


%%%%%%%%%%%%%%%%%%%%%%%%%%%%%%%%%%%%%%%%%%%%%%%%%%%%%%%%%%%%%%%%%%%%%%%%%%%%%%%%%%%%%%%%%%
\begin{frame}
\frametitle{Rotação Geral 3D}

	\begin{block}{Rotação Geral 3D}
		\begin{itemize}
			\item Pode-se rotacionar um objeto em torno de qualquer eixo utilizando algumas rotações e translações.
		\end{itemize}
	\end{block}
	
	\begin{block}{Rotação Geral 3D - Caso Particular}
		\begin{itemize}
			\item Há um caso especial que ocorre quando o eixo de rotação é paralelo a algum eixo de coordenada.
			\item Para obter a rotação desejada é necessário:
				\begin{itemize}
					\item Translade o objeto de forma que o eixo de rotação coincida com o eixo de coordenada.
					\item Executa a rotação.
					\item Translada-se o objeto para posição original.
				\end{itemize}
		\end{itemize}
	\end{block}
	
\end{frame}

%%%%%%%%%%%%%%%%%%%%%%%%%%%%%%%%%%%%%%%%%%%%%%%%%%%%%%%%%%%%%%%%%%%%%%%%%%%%%%%%%%%%%%%%%%
\begin{frame}
\frametitle{Rotação 3D}


	\begin{figure}[!h]
			\begin{center}
			\includegraphics[width=0.7\textwidth]{Figures/RotGer}
			\caption{Rotação em torno de um eixo coincidente com eixo x das coordenadas.}
			\end{center}
	\end{figure}
	
\end{frame}

%%%%%%%%%%%%%%%%%%%%%%%%%%%%%%%%%%%%%%%%%%%%%%%%%%%%%%%%%%%%%%%%%%%%%%%%%%%%%%%%%%%%%%%%%%
\begin{frame}
\frametitle{Rotação 3D}


	\begin{block}{Rotação Geral 3D}
		\begin{itemize}
			\item A sequencia de transformações pode ser dada por:
				$\textbf{P}' = \textbf{T}^{-1} \cdot \textbf{R}_{x}(\theta) \cdot \textbf{T} \cdot \textbf{P}$
				
			\item Ou seja, é a mesma matriz de rotação 2D quando o ponto não coincide com a origem.
		\end{itemize}
	\end{block}
	
\end{frame}

%%%%%%%%%%%%%%%%%%%%%%%%%%%%%%%%%%%%%%%%%%%%%%%%%%%%%%%%%%%%%%%%%%%%%%%%%%%%%%%%%%%%%%%%%%
\begin{frame}
\frametitle{Rotação 3D}


	\begin{block}{Rotação Geral 3D}
		\begin{itemize}
			\item Quando o eixo de rotação não é paralelo a algum eixo de coordenada, algumas transformações adicionais são necessárias.
				\begin{itemize}
					\item São necessárias algumas rotações para alinhar o eixo de rotação com o eixo de coordenada escolhido e para trazer o eixo de rotação a posição original.
				\end{itemize}
		\end{itemize}
	\end{block}
	
\end{frame}

%%%%%%%%%%%%%%%%%%%%%%%%%%%%%%%%%%%%%%%%%%%%%%%%%%%%%%%%%%%%%%%%%%%%%%%%%%%%%%%%%%%%%%%%%%
\begin{frame}
\frametitle{Rotação 3D}


	\begin{block}{Rotação Geral 3D}
		\begin{itemize}
			\item Dado o eixo de rotação e o ângulo de rotação:
				\begin{enumerate}
					\item Translada-se o objeto de modo que o eixo de rotação passe pela origem do sistema.
					\item Rotaciona-se o objeto de modo que o eixo de rotação coincida com um dos eixos de coordenadas.
					\item Realiza-se a rotação em cima do eixo de coordenada escolhido.
					\item Aplica-se a rotação inversa para trazer o eixo de rotação a sua orientação original.
					\item Aplica-se a translação inversa para trazer o eixo de rotação para sua posição inicial.
				\end{enumerate}
		\end{itemize}
	\end{block}
	\begin{block}{Eixo de Coordenada Escolhido para Orientação}
		\begin{itemize}
			\item Por conveniência o eixo de coordenada escolhido para o alinhamento é o eixo $z$.
		\end{itemize}
	\end{block}
	
\end{frame}

%%%%%%%%%%%%%%%%%%%%%%%%%%%%%%%%%%%%%%%%%%%%%%%%%%%%%%%%%%%%%%%%%%%%%%%%%%%%%%%%%%%%%%%%%%
\begin{frame}
\frametitle{Rotação 3D}


	\begin{figure}[!h]
			\begin{center}
			\includegraphics[width=0.8\textwidth]{Figures/RoteEixGeral}
			\end{center}
	\end{figure}
	
\end{frame}


%%%%%%%%%%%%%%%%%%%%%%%%%%%%%%%%%%%%%%%%%%%%%%%%%%%%%%%%%%%%%%%%%%%%%%%%%%%%%%%%%%%%%%%%%%%
%\begin{frame}
%\frametitle{Rotação 3D}
%	\begin{block}{Rotação Geral 3D}
%		\begin{itemize}
%			\item Assumindo que o eixo de rotação é formado por dois pontos $\textbf{P}_1$ e $\textbf{P}_2$ e a rotação é no sentido anti-horário, podemos calcular suas componente da seguinte forma:\\
%			$\textbf{V} = \textbf{P}_2 - \textbf{P}_1 = (x_2 - x_1,y_2 - y_1,z_2 - z_1)$
%			\item E o vetor de tamanho unitário:\\
%				$\textit{u} = \frac{\textbf{V}}{|\textbf{V}|} = (a,b,c)$
%		\end{itemize}
%	\end{block}
%	
%	\begin{figure}[!h]
%			\begin{center}
%			\includegraphics[width=0.3\textwidth]{Figures/PonRot}
%			\end{center}
%	\end{figure}
%	
%\end{frame}
%
%%%%%%%%%%%%%%%%%%%%%%%%%%%%%%%%%%%%%%%%%%%%%%%%%%%%%%%%%%%%%%%%%%%%%%%%%%%%%%%%%%%%%%%%%%%
%\begin{frame}
%\frametitle{Rotação 3D}
%	\begin{block}{Rotação Geral 3D}
%		\begin{itemize}
%			\item O primeiro passo para a efetuar a rotação é em definir uma matriz de translação que translade o eixo de rotação de modo que o mesmo passe pela origem.
%			\begin{itemize}
%				\item Como a rotação é dada no sentido anti-horário, movemos o ponto $\textbf{P}_1$ para a origem.
%				
%				\begin{eqnarray*}
%					\begin{bmatrix}
%						1 & 0 & 0 & -x_1 \\
%						0 & 1 & 0 & -y_1 \\
%						0 & 0 & 1 & -z_1 \\
%						0 & 0 & 0 & 1 \\
%					\end{bmatrix}
%				\end{eqnarray*}
%			\end{itemize}
%		\end{itemize}
%	\end{block}
%	
%	\begin{figure}[!h]
%			\begin{center}
%			\includegraphics[width=0.2\textwidth]{Figures/RotMovOri}
%			\end{center}
%	\end{figure}
%	
%\end{frame}
%
%
%%%%%%%%%%%%%%%%%%%%%%%%%%%%%%%%%%%%%%%%%%%%%%%%%%%%%%%%%%%%%%%%%%%%%%%%%%%%%%%%%%%%%%%%%%%
%\begin{frame}
%\frametitle{Rotação 3D}
%	\begin{block}{Rotação Geral 3D}
%		\begin{itemize}
%			\item Após posicionar o eixo de rotação na origem, temos que encontrar a matriz de transformação que coloca o eixo sobre o eixo $z$.
%			\begin{itemize}
%				\item Existem várias maneiras de efetuar este posicionamento. Um deles é rotacionar o eixo de rotação sobre o eixo $x$, e logo após sobre o eixo $y$.
%				\item A rotação sobre o eixo $x$ define um vetor \textit{u} no plano $xz$.
%				\item A rotação em $y$ rotaciona \textit{u} até sobrepor o eixo $z$.
%			\end{itemize}
%		\end{itemize}
%	\end{block}
%	
%		\begin{figure}[!h]
%			\begin{center}
%			\includegraphics[width=0.6\textwidth]{Figures/RotXZ}
%			\end{center}
%		\end{figure}
%	
%\end{frame}
%
%
%%%%%%%%%%%%%%%%%%%%%%%%%%%%%%%%%%%%%%%%%%%%%%%%%%%%%%%%%%%%%%%%%%%%%%%%%%%%%%%%%%%%%%%%%%%
%\begin{frame}
%\frametitle{Rotação 3D}
%	\begin{block}{Rotação Geral 3D}
%		\begin{itemize}
%			\item A rotação em torno do eixo $x$ pode ser obtida utilizando senos e cossenos do vetor \textit{u} no plano $xz$.
%			\item O ângulo $\theta$ utilizado na rotação é o ângulo obtido da projeção de \textit{u} no plano $yz$ com o eixo $z$ positivo.
%		\end{itemize}
%	\end{block}
%	
%	\begin{figure}[!h]
%			\begin{center}
%			\includegraphics[width=0.4\textwidth]{Figures/Projyz}
%			\end{center}
%	\end{figure}
%	
%\end{frame}
%
%
%%%%%%%%%%%%%%%%%%%%%%%%%%%%%%%%%%%%%%%%%%%%%%%%%%%%%%%%%%%%%%%%%%%%%%%%%%%%%%%%%%%%%%%%%%%
%\begin{frame}
%\frametitle{Rotação 3D}
%	\begin{block}{Rotação Geral 3D}
%		\begin{itemize}
%			\item Se a projeção de \textit{u} no plano $yz$ for $u' = (0,b,c)$, então o cosseno do ângulo de rotação $\theta$ pode ser obtido a partir do produto escalar de $u'$ com o vetor unitário $u$ no eixo $z$:
%		
%			\begin{equation*}
%				cos \theta = \frac{'u \cdot u_z}{|u'| \cdot |u_z|}= \frac{c}{d}
%			\end{equation*}
%			\item Onde $d$ é a magnitude de $u'$, ou seja:
%				\begin{equation}
%					d = \sqrt{b^2+c^2}
%				\end{equation}
%		\end{itemize}
%	\end{block}
%			
%	
%\end{frame}


%%%%%%%%%%%%%%%%%%%%%%%%%%%%%%%%%%%%%%%%%%%%%%%%%%%%%%%%%%%%%%%%%%%%%%%%%%%%%%%%%%%%%%%%%%%
\begin{frame}
\section{Composição de Transformações 3D}
\frametitle{Composição de Transformações 3D}
	\begin{block}{Rotação Geral 3D}
		\begin{itemize}
			\item A Composição de transformações 3D é semelhante a 2D, isto é, são compostas por multiplicações de matrizes.
			\item A transformação mais a direita será a primeira a ser aplicada.
		\end{itemize}
	\end{block}

\end{frame}


%%%%%%%%%%%%%%%%%%%%%%%%%%%%%%%%%%%%%%%%%%%%%%%%%%%%%%%%%%%%%%%%%%%%%%%%%%%%%%%%%%%%%%%%%%%
\begin{frame}
\section{Outras Transformações 3D}
\subsection{Reflexão 3D}
\frametitle{Reflexão 3D}
	\begin{block}{Reflexão 3D}
		\begin{itemize}
			\item A reflexão é semelhante a reflexão 2D, ou seja, é feito a rotação de $180^{\circ}$ graus sobre um eixo (em 3D é um plano) de rotação.
			\item Quando um plano de rotação é um plano coordenado ($xy,xz,yz$), a transformação pode ser vista invertendo a orientação do vetor ortogonal ao plano.
		\end{itemize}
	\end{block}
	
	\begin{figure}[!h]
			\begin{center}
			\includegraphics[width=0.7\textwidth]{Figures/Ref3D}
			\end{center}
	\end{figure}

\end{frame}

%%%%%%%%%%%%%%%%%%%%%%%%%%%%%%%%%%%%%%%%%%%%%%%%%%%%%%%%%%%%%%%%%%%%%%%%%%%%%%%%%%%%%%%%%%%
\begin{frame}
\frametitle{Reflexão 3D}
	\begin{block}{Reflexão 3D}
		\begin{itemize}
			\item Se o plano da reflexão for o plano $xy$, então esta transformação é obtida pela inversão do sinal de $z$ e mantendo as coordenadas $x$ e $y$.
			\begin{equation*}
				\textbf{M}_{z-Reflect}\begin{bmatrix}
					1	&	0	&	0	&	0 \\
					0	&	1	&	0	&	0 \\
					0	&	0	&	-1	&	0 \\
					1	&	0	&	0	&	1 \\
				\end{bmatrix}
			\end{equation*}
			\item As reflexões nos planos $yz$ e $xz$ são obtidas de forma semelhante. 
		\end{itemize}
	\end{block}
\end{frame}

%%%%%%%%%%%%%%%%%%%%%%%%%%%%%%%%%%%%%%%%%%%%%%%%%%%%%%%%%%%%%%%%%%%%%%%%%%%%%%%%%%%%%%%%%%%
\subsection{Cisalhamento 3D}
\begin{frame}
\frametitle{Cisalhamento 3D}
	\begin{block}{Cisalhamento 3D}
		\begin{itemize}
			\item O cisalhamento relativo a $x$ e $y$ em 3D são os mesmos utilizados em 2D, mas agora também é possível faze-lo no eixo $z$. 
			\item Cisalhamento no eixo $x$\\
			\begin{equation*}
				\textbf{S}(sh_{zx},sh_{zy})\begin{bmatrix}
					1	&	0	&	sh_{zx}	&	0 \\
					0	&	1	&	sh_{zy}	&	0 \\
					0	&	0	&	1	&	0 \\
					1	&	0	&	0	&	1 \\
				\end{bmatrix}
			\end{equation*}
			
		\end{itemize}
	\end{block}
\end{frame}

%----------------------------------------------------------------------------------------

\end{document} 