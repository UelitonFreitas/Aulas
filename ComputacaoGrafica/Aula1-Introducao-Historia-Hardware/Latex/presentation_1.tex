%%%%%%%%%%%%%%%%%%%%%%%%%%%%%%%%%%%%%%%%%
% Beamer Presentation
% LaTeX Template
% Version 1.0 (10/11/12)
%
% This template has been downloaded from:
% http://www.LaTeXTemplates.com
%
% License:
% CC BY-NC-SA 3.0 (http://creativecommons.org/licenses/by-nc-sa/3.0/)
%
%%%%%%%%%%%%%%%%%%%%%%%%%%%%%%%%%%%%%%%%%

%----------------------------------------------------------------------------------------
%	PACKAGES AND THEMES
%----------------------------------------------------------------------------------------

\documentclass{beamer}

\mode<presentation> {

% The Beamer class comes with a number of default slide themes
% which change the colors and layouts of slides. Below this is a list
% of all the themes, uncomment each in turn to see what they look like.

%\usetheme{default}
%\usetheme{AnnArbor}
%\usetheme{Antibes}
%\usetheme{Bergen}
%\usetheme{Berkeley}
%\usetheme{Berlin}
%\usetheme{Boadilla}
%\usetheme{CambridgeUS}
%\usetheme{Copenhagen}
\usetheme{Darmstadt}
%\usetheme{Dresden}
%\usetheme{Frankfurt}
%\usetheme{Goettingen}
%\usetheme{Hannover}
%\usetheme{Ilmenau}
%\usetheme{JuanLesPins}
%\usetheme{Luebeck}
%\usetheme{Madrid}
%*\usetheme{Malmoe}
%\usetheme{Marburg}
%\usetheme{Montpellier}
%\usetheme{PaloAlto}
%\usetheme{Pittsburgh}
%\usetheme{Rochester}
%\usetheme{Singapore}
%\usetheme{Szeged}
%\usetheme{Warsaw}

% As well as themes, the Beamer class has a number of color themes
% for any slide theme. Uncomment each of these in turn to see how it
% changes the colors of your current slide theme.

%\usecolortheme{albatross}
%\usecolortheme{beaver}
%\usecolortheme{beetle}
%\usecolortheme{crane}
%\usecolortheme{dolphin}
%\usecolortheme{dove}
%\usecolortheme{fly}
%\usecolortheme{lily}
\usecolortheme{orchid}
%\usecolortheme{rose}
%\usecolortheme{seagull}
%\usecolortheme{seahorse}
%\usecolortheme{whale}
%\usecolortheme{wolverine}

%\setbeamertemplate{footline} % To remove the footer line in all slides uncomment this line
%\setbeamertemplate{footline}[page number] % To replace the footer line in all slides with a simple slide count uncomment this line

%\setbeamertemplate{navigation symbols}{} % To remove the navigation symbols from the bottom of all slides uncomment this line
}


\usepackage{graphicx} % Allows including images
\usepackage{booktabs} % Allows the use of \toprule, \midrule and \bottomrule in tables
\usepackage{xspace}
\usepackage{caption}
\usepackage{subfigure}
\usepackage[english,brazil]{babel}
\usepackage[utf8]{inputenc}

%Renomeia o nome padrao das figuras.
\renewcommand{\figurename}{Figura}
\renewcommand{\tablename}{Tabela}
%----------------------------------------------------------------------------------------
%	TITLE PAGE
%----------------------------------------------------------------------------------------

\title[Computação Gráfica]{Introdução à Computação Gráfica} % The short title appears at the bottom of every slide, the full title is only on the title page

\author{Uéliton Freitas} % Your name
\institute[UFMS] % Your institution as it will appear on the bottom of every slide, may be shorthand to save space
{
Universidade Católica Don Bosco - UCDB \\ % Your institution for the title page
\medskip
\textit{freitas.ueliton@gmail.com} % Your email address
}
\date{\today} % Date, can be changed to a custom date


\begin{document}

\begin{frame}
\titlepage % Print the title page as the first slide
\end{frame}

\begin{frame}
\frametitle{Sumário} % Table of contents slide, comment this block out to remove it
\tableofcontents % Throughout your presentation, if you choose to use \section{} and \subsection{} commands, these will automatically be printed on this slide as an overview of your presentation
\end{frame}




%----------------------------------------------------------------------------------------
%	PRESENTATION SLIDES
%----------------------------------------------------------------------------------------

%------------------------------------------------
\section{Introdução} 
%------------------------------------------------

%\section{Speeded-Up Robust Features - SURF} % A subsection can be created just before a set of slides with a common theme to further break down your presentation into chunks
%\section{Baf Of Features and Colors}

%\section{Refer\^encias}
%%%%%%%%%%%%%%%%%%%%%%%%%%%%%%%%%%%%%%%%%%%%%%%%%%%%%%%%%%%%%%%%%%%%%%%%%%%%%%%%%%%%%%%%%%
\begin{frame}
\frametitle{Introdução}

\begin{itemize}
	\item O que é Computação Gráfica?
	\begin{figure}[htb!]
  		\centering
    			\subfigure[Dota 2]{%
      		\includegraphics[width=0.3\textwidth]{Figures/dk}}\qquad
   			\subfigure[Battle Field 4]{%
     		\includegraphics[width=0.3\textwidth]{Figures/bf4}}\qquad
     		\subfigure[Como Treinar seu Dragão 2]{%
     		\includegraphics[width=0.3\textwidth]{Figures/htd}}\qquad
     		\subfigure[Projeto de um Avião]{%
     		\includegraphics[width=0.3\textwidth]{Figures/plane}}\qquad
  		\caption{}
  		\label{iep}
	\end{figure}

\end{itemize}

	
\end{frame}

%%%%%%%%%%%%%%%%%%%%%%%%%%%%%%%%%%%%%%%%%%%%%%%%%%%%%%%%%%%%%%%%%%%%%%%%%%%%%%%%%%%%%%%%%%

\begin{frame}
\frametitle{Introdução}

\begin{itemize}
	\item O que é Computação Gráfica? \\
	\textbf{Computação Gráfica} é a ciência e arte da comunicação visual via display de um computador e integração dos dispositivos envolvidos.
	\begin{figure}[htb!]
  \centering
    \subfigure[Periféricos utilizados em Computação gráfica.]{\label{dota}%
      \includegraphics[width=0.3\textwidth]{Figures/pc}}\qquad
   \subfigure[Simulador de Voo da NASA.]{\label{bf4}%
     \includegraphics[width=0.3\textwidth]{Figures/nasa}}\qquad
  \caption{}
  \label{iep}
\end{figure}

\end{itemize}
\end{frame}

%%%%%%%%%%%%%%%%%%%%%%%%%%%%%%%%%%%%%%%%%%%%%%%%%%%%%%%%%%%%%%%%%%%%%%%%%%%%%%%%%%%%%%%%%%

\begin{frame}
\frametitle{Introdução}

\begin{block}


\begin{itemize}
	
	\item<1-> O que é Computação Gráfica?\\
		\textbf{Computação Gráfica} é a ciência e arte da comunicação visual via display de um computador e integração dos dispositivos envolvidos.
	\item<2-> O que a Computação Gráfica aborda?\\
		\begin{itemize}
			\item<3-> Técnicas para geração, exibição e manipulação e interpretação de modelos de imagens utilizando o computador.
		\end{itemize}
	\item<3-> Possui vários tipos de usuários:
		\begin{itemize}
			\item Ciência, engenharia, medicina, arte, publicidade, ...
		\end{itemize}
	\item<4-> Mais informações podem ser encontradas \textit{Association for Computing Machinery’s Special Interest Group on Computer Graphics and Interactive Techniques}(http://www.siggraph.org/)

\end{itemize}
\end{block}
\end{frame}

%%%%%%%%%%%%%%%%%%%%%%%%%%%%%%%%%%%%%%%%%%%%%%%%%%%%%%%%%%%%%%%%%%%%%%%%%%%%%%%%%%%%%%%%%%


\section{Evolução da Computação Gráfica}
\begin{frame}
\frametitle{Evolução da Computação Gráfica}

\begin{block}


\begin{itemize}
	\item<1-> Sketchpad - 1963:
	\begin{itemize}
		\item<1-> Ivan Sutherland apresenta um sistema de que desenvolvia em seu Ph.D no MIT.
		\item<1-> Programa de manipulação e criação de elementos 2D em um monitor de vídeo.
		\item<1-> Utilizava uma \textbf{caneta óptica} como dispositivo de entrada e o monitor como dispositivo de saída.
		
		\item<1-> Primeira tentativa de usar dispositivo de vídeo como dispositivo de integração assim como o computador para gerar e exibir figuras.
	\end{itemize}
	

\end{itemize}
\end{block}

\end{frame}

%%%%%%%%%%%%%%%%%%%%%%%%%%%%%%%%%%%%%%%%%%%%%%%%%%%%%%%%%%%%%%%%%%%%%%%%%%%%%%%%%%%%%%%%%%


\begin{frame}
\frametitle{Evolução da Computação Gráfica}
	\begin{figure}[!h]
			\begin{center}
			\includegraphics[width=0.5\textwidth]{Figures/ivan}
			\caption{Ivan Sutherland no console TX-2.}\label{ivan}
			\end{center}
	\end{figure}
\end{frame}

%%%%%%%%%%%%%%%%%%%%%%%%%%%%%%%%%%%%%%%%%%%%%%%%%%%%%%%%%%%%%%%%%%%%%%%%%%%%%%%%%%%%%%%%%%


\begin{frame}
\frametitle{Evolução da Computação Gráfica}

\begin{block}

	\begin{itemize}
		\item<1-> Dispositivos de Exibição:
		\begin{itemize}
			\item<1-> Natureza analógica:vector graphics.
			\item<1-> Os desenhos eram formados por segmentos de retas.
			\item<1-> Tecnologia cara e sem cores.
		\end{itemize}
		\item<1-> Primeiros programas CAD.
		\item<1-> Pouca iteração com o usuário e a tecnologia como um tudo era muito cara.
	
	\end{itemize}
\end{block}

\end{frame}

%%%%%%%%%%%%%%%%%%%%%%%%%%%%%%%%%%%%%%%%%%%%%%%%%%%%%%%%%%%%%%%%%%%%%%%%%%%%%%%%%%%%%%%%%%


\begin{frame}
\frametitle{Evolução da Computação Gráfica}

	\begin{figure}[!h]
		\begin{center}
			\includegraphics[width=0.9\textwidth]{Figures/crt}
			\caption{CRT.}
		\end{center}
		
	\end{figure}

\end{frame}

%%%%%%%%%%%%%%%%%%%%%%%%%%%%%%%%%%%%%%%%%%%%%%%%%%%%%%%%%%%%%%%%%%%%%%%%%%%%%%%%%%%%%%%%%%


\begin{frame}
\frametitle{Evolução da Computação Gráfica}

\begin{block}

	\begin{itemize}
		\item<1-> Década de 70:
		\begin{itemize}
			\item<1-> Disseminação de aplicativos.
			\item<1-> Evolução da Computação Gráfica de \textit{hardware}.
				\begin{itemize}
					\item Surgimento da \textbf{tecnologia matricial} (\textit{raster graphics}).
					\item Imagens formadas por matrizes de pontos ou \textit{pixels}.
					\item Tecnologia mais \textbf{barata}.
					\item Possibilita o uso de cores e preenchimento das figuras.
					\item \textbf{Aliasing}.
				\end{itemize}
			\item Aumento da capacidade gráfica.
			\item Melhores dispositivos de integração (Mouse 1968).
			\item Novos paradigmas em IHC (criação de janelas).
		\end{itemize}
	\end{itemize}
\end{block}

\end{frame}


%%%%%%%%%%%%%%%%%%%%%%%%%%%%%%%%%%%%%%%%%%%%%%%%%%%%%%%%%%%%%%%%%%%%%%%%%%%%%%%%%%%%%%%%%%


\begin{frame}
\frametitle{Evolução da Computação Gráfica}

	\begin{figure}[!h]
		\begin{center}
			\includegraphics[width=0.9\textwidth]{Figures/matrixcrt}
			\caption{CRT Matricial.}
		\end{center}
		
	\end{figure}

\end{frame}

%%%%%%%%%%%%%%%%%%%%%%%%%%%%%%%%%%%%%%%%%%%%%%%%%%%%%%%%%%%%%%%%%%%%%%%%%%%%%%%%%%%%%%%%%%


\begin{frame}
\frametitle{Evolução da Computação Gráfica}

\begin{block} {Pixel}
	O \textbf{Pixel} é uma pequena área da imagem armazenada no \textbf{Frame Buffer}.
	
\end{block}

	\begin{figure}[!h]
		\begin{center}
			\subfigure[Imagem original.]{
      			\includegraphics[width=0.6\textwidth]{Figures/ironman}}\qquad
      		\subfigure[Imagen aumentada.]{
      			\includegraphics[width=0.3\textwidth]{Figures/tumb}}\qquad
		\end{center}
		
	\end{figure}

\end{frame}

%%%%%%%%%%%%%%%%%%%%%%%%%%%%%%%%%%%%%%%%%%%%%%%%%%%%%%%%%%%%%%%%%%%%%%%%%%%%%%%%%%%%%%%%%%


\begin{frame}
\frametitle{Evolução da Computação Gráfica}

\begin{block} {Pixel}
	O \textbf{Pixel} é uma pequena área da imagem armazenada no \textbf{Frame Buffer}.
	
\end{block}

	\begin{figure}[!h]
		\begin{center}
			\includegraphics[width=0.8\textwidth]{Figures/framebuffer}
			\caption{Representação do Frame Buffer.}
		\end{center}
		
	\end{figure}

\end{frame}

%%%%%%%%%%%%%%%%%%%%%%%%%%%%%%%%%%%%%%%%%%%%%%%%%%%%%%%%%%%%%%%%%%%%%%%%%%%%%%%%%%%%%%%%%%


\begin{frame}
\frametitle{Evolução da Computação Gráfica}

\begin{block}

	\begin{itemize}
		\item<1-> Década de 80:
		\begin{itemize}
			\item<1-> Pacotes Gráficos.
				\begin{itemize}
					\item \textbf{Portabilidade} (Independência de dispositivos).
					\item Reutilização.
					\item API's: OpenGL. Aplicativos independentes de SO (sistemas de janelas etc.).
				\end{itemize}
			\item \textbf{Computação Gráfica 3D}
				\begin{itemize}
					\item Representação dos objetos no espaço 3D.
				\end{itemize}
			
		\end{itemize}
	\end{itemize}
\end{block}

\end{frame}


%%%%%%%%%%%%%%%%%%%%%%%%%%%%%%%%%%%%%%%%%%%%%%%%%%%%%%%%%%%%%%%%%%%%%%%%%%%%%%%%%%%%%%%%%%


\begin{frame}
\frametitle{Evolução da Computação Gráfica}

	\begin{figure}[!h]
		\begin{center}
			\includegraphics[width=0.8\textwidth]{Figures/ModelagemCG}
			\caption{Sistema Gráfico.}
		\end{center}
		
	\end{figure}

\end{frame}

%%%%%%%%%%%%%%%%%%%%%%%%%%%%%%%%%%%%%%%%%%%%%%%%%%%%%%%%%%%%%%%%%%%%%%%%%%%%%%%%%%%%%%%%%%


\begin{frame}
\frametitle{Evolução da Computação Gráfica}

\begin{block}

	\begin{itemize}
		\item<1-> Técnicas de criação de mundo 3D:
		\begin{itemize}
			\item \textbf{Modelagem} - Criação da representação de um objeto.
				\begin{itemize}
					\item Informações geométricas.
					\item Informações sobre as fontes de luz e observador.
					\item Informações sobre os materiais do objeto.
					\item Poligonização: Aproximação de uma forma de um objeto por meio de uma malha de faces poligonais (como triângulos). 
				\end{itemize}
			\item \textbf{Renderização} e Animação - Meios de se exibir o objeto 
				\begin{itemize}
					\item Geração de uma imagem a partir dos modelos.
					\item Simulações da iteração de fontes de luz.
				\end{itemize}
			
		\end{itemize}
	\end{itemize}
\end{block}

\end{frame}


%%%%%%%%%%%%%%%%%%%%%%%%%%%%%%%%%%%%%%%%%%%%%%%%%%%%%%%%%%%%%%%%%%%%%%%%%%%%%%%%%%%%%%%%%%


\begin{frame}
\frametitle{Evolução da Computação Gráfica}

\begin{block}

	\begin{itemize}
		\item<1-> Década de 90:
		\begin{itemize}
			\item Gama de técnicas em síntese de imagens.
				\begin{itemize}
					\item Estratégias clássicas de modelagem: fronteiras, CSG, octrees...
					\item Estratégias para descrições de modelos: varreduras, formulações matemáticas para definições alternativas para curvas e superfícies.
					\item Estratégias alternativas de modelagens: fractais, partículas..
					\item Estratégias de rendering sofisticadas: ray tracing, mapeamento de textura. 
				\end{itemize}
			\item As áreas relacionadas amadureceram.
			
		\end{itemize}
	\end{itemize}
\end{block}

\end{frame}


%%%%%%%%%%%%%%%%%%%%%%%%%%%%%%%%%%%%%%%%%%%%%%%%%%%%%%%%%%%%%%%%%%%%%%%%%%%%%%%%%%%%%%%%%%


\begin{frame}
\frametitle{Evolução da Computação Gráfica}

	\begin{figure}[!h]
		\begin{center}
			\includegraphics[width=0.8\textwidth]{Figures/graphicPipeline}
			\caption{Pipeline Gráfico.}
		\end{center}
		
	\end{figure}

\end{frame}
%----------------------------------------------------------------------------------------

\end{document} 